\chapter{GIS in Archeologia}

\emph{Il seguente testo è tratto dalla voce di Wikipedia inglese ``GIS in Archaeology''\cite{wiki:xxx}, e fornisce una rapida overview sul rapporto oggi esistente tra sistemi informativi geografici e metodi della ricerca archeologica, mettendo in evidenza i campi in cui sono stati collezionati evidenti successi e le problematiche relative ai futuri sviluppi della collaborazione tra le due discipline.}\\

I GIS o \emph{Geographic Information Systems} sono stati negli ultimi 10 anni un importante strumento in archeologia. Effettivamente, gli archeologi furono tra i primi ad adottare, sia come utenti che come sviluppatori i GIS e la \emph{GIScience}, ovvero la \emph{Geographic Information Science}. La combinazione di GIS ed archeologia è stata considerata perfetta in quanto l'archeologia è lo studio della dimensione spaziale del comportamento umano nel tempo, e qualsiasi informazione archeologica si accompagna ad una componente spaziale.

Dato che l'archeologia punta a svelare gli eventi storici attraverso lo studio della geografia, del tempo e della cultura, i risultati degli studi archeologici sono ricchi di informazioni spaziali. Il GIS è il sistema consacrato all'elaborazione di queste moli di dati, specialmente di quelli georiferiti. Riesce anche ad essere uno strumento economico, accurato e veloce. Le funzioni rese disponibili tramite il GIS aiutano a collezionare dati, ad archiviarli e reperirli, a manipolarli anche per necessità specifiche e, infine, a mostrarli in maniera tale da essere visivamente comprensibili per l'utente.  L'aspetto più importante del GIS in archeologia risiede, comunque, non nel suo uso come sistema puro per creare mappe, ma nella sua capacità di mescolare ed analizzare diversi tipi di dati, in maniera da dare vita ad un'informazione
che non sia necessariamente equivalente alla somma delle informazioni introdotte. L'uso del GIS in archeologia ha cambiato non solo la maniera in cui gli archeologi acquisiscono e gestiscono i dati, ma anche la maniera in cui questi pensano lo spazio. Il GIS è quindi diventato una scienza, esulando dalla concezione di semplice strumento.


\section{Il GIS nel rilievo}
	Rilievo e documentazione sono importanti nell'ottica di preservare i beni culturali, e l'archeologia unita al GIS rende la ricerca ed il lavoro sul campo efficienti e precisi. La ricerca condotta usando le potenzialità del GIS è usata come strumento decisionale per prevenire importanti perdite d'informazione che potrebbero avere un forte impatto sullo studio archeologico. Il GIS si dimostra in questo contesto un utile strumento per la programmazione del territorio e per la gestione delle sue risorse culturali, riuscendo a fornire informazioni sulle risorse valorizzabili, attraverso l'acquisizione e la manutenzione di dati rilevati sui siti storici.

	In archeologia, il GIS aumenta l'abilità di mappare e registrare i dati quando è usato direttamente sul campo di scavo. Ciò consente di avere immediato accesso ai dati raccolit per analisi e visualizzazioni a livello di studio isolato, a livelli più ampi, incorporandoli con altre importanti fonti di dati per meglio comprendere il sito e i ritrovamenti.

	L'utilità del GIS è estesa alla capacità di modellare e prevedere come i siti archeologici saranno usati dalle compagnie coinvolte nell'utilizzo di grandi estensioni di risorse territoriali (come ad esempio il Dipartimento dei Trasporti). Negli Stati Uniti, la sezione 106 del \emph{National Preservation Act} richiede specificatamente che siano effettuati studi d'impatto delle opere sui siti storici, attraverso progetti finanziati dalla federazione. L'utilizzo di GIS per accertare l'esistenza di siti archeologici può essere identificata attraverso il \emph{predictive modeling}, la modellazione preventiva. Questi studi -- ed i successivi risultati -- sono utilizzati da chi gestisce il territorio per orientare importanti decisioni e pianificare gli sviluppi futuri. Il GIS rende questi processi più veloci e precisi.

	Esistono diversi processi e funzionalità nel GIS impiegate in ambito archeologico. L'analisi spaziale \emph{intrasito} o l'analisi della distribuzione delle informazioni sul sito aiutano a comprendere la formazione, e il cambiamento della documentazione archeologica del sito. Ciò conduce ad una ricerca, a delle analisi ed a trarre conclusioni.  I vecchi metodi utilizzati a questo scopo fornivano limitata esposizione al sito e risultati molto deludenti. Il \emph{predictive modeling} è utilizzato attraverso l'acquisizione di dati come l'idrografia e l'ipsografia per sviluppare modelli che affianchino i dati archeologici, portando a risultati migliori. I dati puntuali nel GIS sono usati per focalizzare l'attenzione su un luogo e calcolare le tendenze nei dati registrati, o per interpolare dati sparsi e non riconducibili ad alcuna tendenza. La mappatura della densità è effettuata per l'analisi delle tendenze di un certo luogo e l'interpolazione è finalizzata ad aiutare il processo di studio dei reperti in superficie attraverso la creazione di superfici, partendo da dati puntuali e allo scopo di trovare le zone occupate nel sito. I dati relativi alle aree sono usati ancora più frequentemente e, focalizzandosi sul paesaggio e sulla regione aiutano ad interpretare i siti archeologici nella loro impostazione e in relazione al loro contesto. I dati relativi alle aree sono analizzati attraverso la modellazione di previsioni, che può arrivare a prevedere le aree di occupazione in una certa regione.  Tale processo si basa sulla conoscenza a disposizione attualmente, sui metodi di previsione e sui risultati di scavi condotti nella regione stessa. È un modello di previsione usato in primo luogo nella gestione delle risorse culturali.


\section{Il GIS nell'analisi}
	I GIS offrono la possibilità di conservare, manipolare e combinare molti set di dati, dando luogo a complesse analisi e delineando una certa quantità di possibili panorami. Ad esempio, l'analisi dei bacini (\emph{catchment analysis}) è lo studio dei bacini d'utenza, che riesce a calcolare l'accessibilità di alcune regioni circostanti un certo sito archeologico, dati un certo impego di tempo o di sforzo umano.  Tale studio viene affiancato dall'\emph{analisi visiva}, che si prefigge di definire quali siano le regioni circostanti il sito visibili dal sito stesso. Queste analisi sono state usate per interpretare la relazione tra certi siti ed il loro panorama sociale. La simulazione è un rappresentazoine semplificata della realtà, che tenta di modellare i fenomeni identificando le variabili chiave e le loro interazioni. Tale metodologia è usata per attraversare trasversalmente il problema, come mezzo per testare ipotesi e per generare dati utili in future analisi.

	Negli ultimi anni, è diventato chiaro che gli archeologi saranno capaci di sfruttare a pieno le potenzialità del GIS o di ogni altra tecnologia spaziale se prenderanno coscienza degli specifici potenziali e rischi legati ai dati archeologici e al processo di ricerca. La scienza dell'archeoinformazione cerca di scoprire ed esplorare gli schemi spaziali e temporali e le loro proprietà in ambito archeologico. La ricerca attraverso un approccio esclusivamente archeologico alla gestione dell'informazione produce metodi quantitativi e software specificatamente orientati alla soluzione e comprensione di problemi archeologici.
