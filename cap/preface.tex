\chapter*{Prefazione}

	Questo manualetto nasce dall'esigenza, interna alla Cattedra di Archeologia Medievale dell'Università degli Studi di Bari, di affrontare il problema della documentazione archeologica degli scavi del sito medievale di Siponto (FG).\\

	La progressiva informatizzazione di tutta la documentazione archeologica collezionata in anni di attività ha portato alla necessità di affrontare la gestione del dato elettronico in maniera quanto più possibile flessibile ed economica allo stesso tempo. In questo contesto, GRASS GIS rivela la sua estrema utilità. Una delle caratteristiche del software che gli archeologi apprezzeranno maggiormente è sicuramente la possibilità di gestire una quantità enorme di dati, con un'avidità di potenza di calcolo decisamente ridotta, al contrario di altri software con caratteristiche affini, rilasciati in licenza libera e non (ArchView, Quantum GIS, ecc.).

	La flessibilità di GRASS GIS permette di gestire un numero impressionante di differenti formati di dati, e di creare complesse elaborazioni, pur tuttavia passando per una gestione del dato non eccessivamente complessa. A differenza di altri sistemi GIS, GRASS ha una curva d'apprendimento molto più ripida rispetto ad altri software, situazione che tuttavia è indice di una eccezionale potenza intrinseca, derivante dai circa 400 moduli eseguibili che lo compongono, e dei risultati professionali che si possono ottenere.

	Questo manualetto è orientato ad un pubblico che ha già acquisito una certa familiarità con almeno un sistema GIS, e cerca di affrontare le problematiche più importanti nella gestione del dato archeologico.  L'approccio di GRASS GIS alla risoluzione di alcuni problemi può alle volte sembrare inutilmente macchinoso, ma si scoprirà essere sorprendentemente pratico e veloce una volta acquisita familiarità con la maniera di operare caratteristica del software; soprattutto, GRASS si rivela estremamente utile quando si ha necessità di operare in maniera precisa e consapevole su grandi quantità di dati, come può avvenire durante (o dopo) uno scavo archeologico.

	In questa sede, soprattutto per la grande quantità di documentazione disponibile sui sistemi GIS e su GRASS in particolare, si eviteranno ripetizioni di materiale già pubblicato da altri autori, limitando molti argomenti ad un semplice richiamo. Al termine del manuale, nella Bibliografia, sono disponibili tutti i riferimenti per approfondire le caratteristiche dei sistemi GIS, dell'analisi geografica e territoriale, e sull'utilizzo di GRASS.

	Inoltre, particolare da non trascurare soprattutto in ambito didattico, GRASS GIS è rilasciato in licenza libera (GNU GPL): ciò consente non solo di poter ottenere il software gratuitamente (scaricandolo liberamente dalla rete internet) ma anche di utilizzarlo e di aggiornarlo senza alcun costo di licenza, così come di distribuirlo senza alcun onere a tutti gli studenti. Inoltre, la completa documentazione disponibile online, a cura dello stesso team di sviluppo del software, permette a tutti di avere accesso alle spiegazioni dettagliate sull'utilizzo dei moduli che compongono il programma (senza la necessità di acquisto di manuali aggiuntivi). In ultima analisi, gli utenti più esperti hanno la possibilità di ``estendere'' il software con lo sviluppo di moduli e script personalizzati, per agevolare alcune operazioni ripetitive o per adattare il software di analisi geospaziale alle proprie peculiari esigenze (sono note applicazioni che vanno dall'ingegneria delle reti idriche a quella delle reti stradali, passando per le analisi meteorologiche). Inoltre, GRASS può essere incorporato all'interno di altri programmi che possono fungere da interfaccia utente come ad esempio, Quantum GIS, rispetto al quale ha anche il vantaggio di gestire in un unico layer sia punti che linee che poligoni, facilitando il lavoro di gestione del dato archeologico.\\

	L'utilizzo di GRASS GIS in ambito archeologico segna un ulteriore passo avanti sulla strada della ricerca di standard all'interno della documentazione archeologica, ricerca che si è affermata di pari passo con l'introduzione dell'\emph{archeoinformatica}, disciplina tesa all'applicazione delle metodologie e delle strumentazioni informatiche all'archeologia stratigrafica e all'archeologia quantitativa (in quest'ultimo contesto, GRASS offre potenti strumenti di analisi statistica). Quindi, se a livello di documentazione di scavo non esiste attualmente veramente nessuno standard informatico, certamente per tutto quello che riguarda l'archeologia dei paesaggi e il telerilevamento GRASS è al passo con qualunque altro software proprietario nella gestione e nell'analisi di immagini aeree, dati LIDAR, \emph{et similia}.

	L'adozione di standard, in tutti i campi della ricerca scientifica, è un concreto aiuto alla comunicazione e divulgazione dei risultati.  Tuttavia, la ricerca di standard internazionali dovrebbe passare sia per l'utilizzo di standard aperti, sia attraverso l'impiego di software open source: in entrambe queste affermazioni risiede la coscienza di fondo che la ricerca storica -- ed in particolare quella archeologica -- non può essere appannaggio di pochi, ma deve innescare un processo che vada dallo studio delle fonti, alla ricerca sul campo, all'analisi dei risultati; il processo deve concludersi con una divulgazione che consenta all'Uomo di conoscere e comprendere la propria Storia e quella del luogo in cui vive, per sviluppare una nuova cultura della comprensione e trasmissione al futuro del patrimonio storico.\\

	L'utilizzo di GRASS in archeologia\footnote{Si faccia ad esempio riferimento al \href{http://grass.osgeo.org/wiki/Archeology}{wiki ufficiale di GRASS}.}, al 2009, può vantare innumerevoli campi d'applicazione, spesso mai raggiunti da software proprietari: si guardi ad esempio il lavoro presentato al GRASS Meeting del 2006 da parte di Arc-Team s.n.c.\footnote{Reperibile da \href{file:http://www.dimset.unige.it/eventi/grass/presentazioni/sessione 3/bezzi et al.pdf}{http://www.dimset.unige.it/eventi/grass/presentazioni/sessione 3/bezzi et al.pdf}.} sull'utilizzo dei \emph{voxel} (raster 3D) per la gestione di dati archeologici di scavo, poi ripresi sempre da Arc-Team e da Undine Lieberwirth (quest'ultima al workshop "Archaeologie~und~Computer" di Vienna del 2006\footnote{\emph{Workshop 11 Archaeologie und Computer, Kulturelles Erbe und Neue Technologien} - Wien, Aramus 2006, la prima spedizione archeologica internazionale ad utilizzare soltanto Software Libero.}). I lavori di Emanuel Demetrescu sull'archeologia urbana di Roma\footnote{Le diapositive dell'intervento sono disponibili presso il \href{file:http://www.perseo.lettere.unipd.it/workshop08/lib/exe/fetch.php?id=download&cache=cache&media=workshop08:documenti:demetrescu_lic.odp}{portale} dell'Università degli Studi di Padova.} sono certamente uno degli esempi più sorprendenti delle grandi potenzialità di GRASS (e dei software connessi) applicati ad un caso di archeologia reale, tra l'altro con risultati interessanti anche al di là delle sperimentazioni tecniche. Infine, del team di sviluppo di GRASS fa parte Michael Barton, archeologo/antropologo statunitense che svolge ricerche territoriali sia oltreoceano sia in Spagna. A livello italiano, risulta inoltre molto preziosa la testimonianza offerta nel secondo volume della collana ``Metodi e temi dell'archeologia medievale'', ampiamente descritta nel volume ``Informatica ed Archeologia Medievale -- L'esperienza senese'' (esperienza che ha previsto anche l'utilizzo dei software open source GRASS e QuantumGIS)\footnote{\emph{Metodi e temi dell'archeologia medievale}, vol. 2, ``Informatica e Archeologia Medievale --- L'esperienza senese'', a cura di Vittorio Fronza, Alessandra Nardini, Marco Valenti.}.\\

	Un ringraziamento va a tutte le persone che hanno mostrato interesse e dedicato tempo ed energie alla sperimentazione di software libero in ambito archeologico, in particolare al Dott.~Austacio~Busto ed alla Dott.ssa~Raffaella~Palombella; un enorme ringraziamento è rivolto a Paola~Monno, prima lettrice del manuale, per i suoi preziosi consigli; si ringraziano inoltre per la preziosa collaborazione: Stefano Costa, Luca Delucchi, Luca Mandolesi, Francesco Lovergine. Ovviamente qualsiasi consiglio, considerazione o critica sono benaccetti, all'indirizzo \href{mailto:fradeve11@gmail.com}{fradeve11@gmail.com}.
	\vfill
	La prima versione di questo documento è stata realizzata nel contesto del progetto ``Siponto Aperta'', lanciato dall'associazione culturale O.I.A. -- Open Idea for Archaeology, progetto n. 528 vincitore dell'edizione 2010 del concorso ``Bollenti Spiriti -- Principi Attivi''.\\~\\
	\includegraphics[width=0.4\textwidth]{img/logo-pa/c_Logo_vincitore_orizz}\hfill
	\includegraphics[width=0.2\textwidth]{img/logo-pa/b_Logo_Regione}
