\newcommand\BackgroundPicA{
	\put(140,220){{
		\begin{tikzpicture}
		\node[opacity=0.5]{\includegraphics[scale=5.5]{logo-book}};
		\end{tikzpicture}
}}}

\newcommand*{\titleTH}{\begingroup
	\raggedleft
	\definecolor{shadecolor}{gray}{0.75}
	\vspace*{\baselineskip}
	\hrule \vspace*{0.5cm}
	{\bfseries \Large \fontsize{8mm}{10mm}\selectfont Introduzione all'utilizzo di}\\[\baselineskip]
	{\textcolor{red}{\Huge \fontsize{15mm}{17mm}\selectfont GRASS GIS}}\makebox[0in][l] {\scalebox{1}[-1]{\textcolor{shadecolor}{{\large \fontsize{8mm}{10mm}\selectfont{\textsl{ in Archeologia}}}}}}{\textcolor{red}{\large \fontsize{8mm}{10mm}\selectfont{\textsl{ in Archeologia}}}}\par
	\vspace{10pt}
	\AddToShipoutPicture*{\BackgroundPicA}
	\vfill
	\includegraphics[width=0.1\textwidth]{img/oia-logo}\par
	\vspace{10pt}
	{\Large O.I.A.}\par
	{\normalsize Open Idea for Archaeology}\\
	{\footnotesize Francesco de Virgilio and the\\grass-arch project contributors}\par
	\vspace*{3\baselineskip}
\endgroup}
\thispagestyle{empty}
\titleTH

\pagebreak{}

\pagenumbering{roman}~

\pagebreak{}

\vspace*{4cm}


\epigraph{«Sed omnia praeclara tam difficilia, quam rara sunt.»\\ \medskip{}Tutte le cose eccellenti sono tanto difficili, quanto rare.}{\emph{Etica, De potentia intellectus seu de libertate humana, Propositio XLII, scholium}\\ \textsc{Baruch Spinoza}}
